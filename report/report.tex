\documentclass[a4paper,12pt]{article}
\usepackage[latin1]{inputenc}
\usepackage[T1]{fontenc}
\usepackage{lmodern}
%\usepackage[frenchb]{babel}
\usepackage{fancyhdr}
\usepackage[top=3cm, bottom=2.5cm, left=2.5cm, right=2.5cm]{geometry}
\usepackage{tabto}
\usepackage{eurosym}%pour le oe
\usepackage{amsmath}%pour les matrices
\usepackage{amsfonts}
\usepackage{graphicx}
\usepackage{subfigure}



\usepackage{titling}
\date{\today}




\fancyfoot[L]{\textbf{Pierre Hubert-Brierre}}
\fancyfoot[C]{\textbf{Fac Lyon 1}}
\fancyfoot[R]{\textbf{\thedate~~~~\thepage}}

\begin{document}

\pagestyle{fancy}
\begin{center}
\textbf{\huge Report Implicit Surface}
\end{center}

\pagestyle{fancy}

% \begin{figure}[h]
%   \centering
%   \includegraphics[scale=1]{rec3-400.png}{}
%   % \includegraphics[scale=0.7]{rec3_-_800.png}{}
% \end{figure}

\begin{figure}[!h]
\null\hfill
\subfigure[barkless tree\label{tree-pres}]{\includegraphics[scale=0.3]{tree_images/tree-rec3.png}{}}
\hfill
\subfigure[barked tree\label{bark-pres}]{\includegraphics[scale=0.3]{tree_images/bark-rec3.png}{}}
\hfill
\end{figure}



\section*{What I implemented}

\paragraph{} Object of arity 0:
\begin{itemize}
	\item A sphere of radius $r$ and center $c$.
	\item A capsule of skeleton defined by the segment between $a$ and $b$ and a radius $r$.
\end{itemize}

\paragraph{} Object of arity 2:	

\begin{itemize}
	\item Standards union, intersection and difference operators.
	\item Smooth union, intersection and difference operators.
\end{itemize}

\paragraph{} Object of arity 1:
\begin{itemize}
	\item The translation by a vector $v$.
	\item The rotation by an angle $\theta$ around an axis $\omega$.
	\item The replication of $n$ copie around an axis $\omega$ (explained later).
	\item The simplification (explained in anexe).
	\item The surface noise\cite{lagae2009procedural} (explained in anexe).
\end{itemize}

\paragraph{} For each of those objects, I automatically compute a bounding box (use for the meshing of the object), a bounding sphere (use for the simplification) and a skeleton direction (use for the surface noise).

\section*{How I have implemented}


\subsection*{Bounding volumes and skeleton direction}

\paragraph{} To compute the bounding boxes, bounding spheres and the skeleton direction, I walk on my tree, computing the data of a node using the data of its children.

\paragraph{Bounding Box} The bounding box of a sphere or a capsule is trivial to compute. For operators, I simply take the bounding box of the multiple bounding box of the childrens.

\paragraph{Bousing sphere} Like the bounding box, it's easy to compute the bounding sphere of a sphere or a capsule. For operators, I simply take the bounding sphere of the multiple bounding sphere of the childrens.


\paragraph{Skeleton direction} Unlike the two previous paragraphs, the skeleton direction of a sphere is not defined. I choose an arbitrary unitary direction for the skeleton direction of the sphere. For operators I compute a relevant skeleton direction based on the skeleton direction of childrens.




\subsection*{The replication}

\paragraph{} The replication of $n$ copie of the children around an axis $\omega$ work as follow:
 first compute the BSF to the children, then rotate the children of an angle $\frac{2\pi}{n}$ around the axis $\omega$. Repeat until you have made a whole rotation and return the \texttt{min} of all the values.


\subsection*{The simplification}

\paragraph{} The simplification operator does not modify the represented object, it just helps to compute the BSF of far away points faster. To achieve that, it returns the distance to the bounding sphere for far away points. The details and proof of lipschitzianity if given in annexe.


\subsection*{The surface noise\cite{zanni2013modelisation}}

\paragraph{} It applies a Gabor noise\cite{lagae2009procedural} at the surface of the object to create small random variation on the surface. To compute the surface noise, I project my point on the surface and compute the gabor noise in a local fram computed using the skeleton direction and the normal at that point. More details are given in anexe.


\newpage
\section*{The ternary tree}

% \begin{figure}[!h]
% \null\hfill
% \subfigure[Barked tree with depth of 3\label{fig1a}]{%
%      \includegraphics[scale=1]{ecorsse_rec3-200.png}{}}
%      % \includegraphics[scale=0.6]{ecorsse_rec3-400.png}{}}
%      % \includegraphics[scale=0.3]{ecorsse_rec3_-_800.png}{}}
% \hfill
% \subfigure[Barkless tree with depth of 7\label{fig1b}]{%
%      \includegraphics[scale=1]{rec7-200.png}{}}
%      % \includegraphics[scale=0.6]{rec7-400.png}{}}
%      % \includegraphics[scale=0.25]{rec7_-_800.png}{}}
% \hfill\null
% % \caption{Une figure divisee en deux}\label{fig1}
% \end{figure}
\paragraph{} Here are some picture of my tree:
\begin{figure}[!h]
\null\hfill
\subfigure[n=0\label{tree0}]{\includegraphics[scale=0.6]{tree_images/downscale/tree-rec0.png}{}}
\hfill
\subfigure[n=1\label{tree1}]{\includegraphics[scale=0.6]{tree_images/downscale/tree-rec1.png}{}}
\hfill
\subfigure[n=2\label{tree2}]{\includegraphics[scale=0.6]{tree_images/downscale/tree-rec2.png}{}}
\hfill\null
\end{figure}
\begin{figure}[!h]
\null\hfill
\subfigure[n=3\label{tree3}]{\includegraphics[scale=0.6]{tree_images/downscale/tree-rec3.png}{}}
\hfill
\subfigure[n=4\label{tree4}]{\includegraphics[scale=0.6]{tree_images/downscale/tree-rec4.png}{}}
\hfill
\subfigure[n=6\label{tree6}]{\includegraphics[scale=0.6]{tree_images/downscale/tree-rec5.png}{}}
\hfill\null
\caption{Different barkless tree with a recursion level $n$}
\end{figure}

\begin{figure}[!h]
\null\hfill
\subfigure[n=0\label{bark0}]{\includegraphics[scale=0.6]{tree_images/downscale/bark-rec0.png}{}}
\hfill
\subfigure[n=1\label{bark1}]{\includegraphics[scale=0.6]{tree_images/downscale/bark-rec1.png}{}}
\hfill
\subfigure[n=2\label{bark2}]{\includegraphics[scale=0.6]{tree_images/downscale/bark-rec2.png}{}}
\hfill\null
\end{figure}
\begin{figure}[!h]
\null\hfill
\subfigure[n=3\label{bark3}]{\includegraphics[scale=0.6]{tree_images/downscale/bark-rec3.png}{}}
\hfill
\subfigure[n=4\label{bark4}]{\includegraphics[scale=0.6]{tree_images/downscale/bark-rec4.png}{}}
\hfill
\subfigure[n=6\label{bark6}]{\includegraphics[scale=0.6]{tree_images/downscale/bark-rec5.png}{}}
\hfill\null
\caption{Different barked tree with a recursion level $n$}
\end{figure}

\paragraph{} The object that I choose to represent is a recursive tree, made of a capsule and three other trees. I also added on the tree some bark modelised with a surface noise. Because the noise is very costly to compute, the recursion level is only of 3.

\paragraph{} More specifically, the tree is either:
\begin{itemize}
	\item one \texttt{Capsule} (for the last recursive level)
	\item A smooth union of a \texttt{Capsule} and a \texttt{replication of a translation of a rotation of a simplification of a tree}
\end{itemize}

For the noise, I choose a gabor noise with an averadge of 30 point per unit cube and a projecting direction of (-1, -2, 0) in the local frame (normal, bitangeant, tangeant).




\section*{Anexe}

\subsection*{Simplification details}

\paragraph{} Let consider an object represented by a BSF $f$ and bounding by a sphere of centre $c$ and radius $r$. Then, the simplification of this object is a new BSF $f_k$ (with $k>1$) defined by:


\[
f_k(p) = (1-\alpha_k(p))f(p) + \alpha_k(p)f_S(p)
\]

with
\[
\alpha_k(p) = \left\{\begin{array}{lll}
	0 & \mbox{if } \left\Vert p - c \right\Vert < r \\
	\frac{\left\Vert p - c \right\Vert - r}{(k-1)r} & \mbox{if } r < \left\Vert p - c \right\Vert < kr \\
	1 & \mbox{else.}
	\end{array}
	\right.
\]

and
\[
f_S(p) = \left\Vert p - c \right\Vert - r
\]

\subsection*{The surface is the same : $f(p)=0 \Leftrightarrow f_k(p)=0$}
\paragraph{} The object is still the same :
\begin{itemize}
\item $f(p)=0 \Rightarrow f_k(p)=0$
\item $f_k(p)=0 \Rightarrow f(p)=0$ Because $\alpha_k(p)f_S(p) \ge 0$ and $\alpha_k(p)f_S(p) = 0 \Rightarrow p \in B(c, r)$
\end{itemize}

\subsection*{$f_k$ is $1 + \frac{k+2}{k-1}$ lipschitz.}
\paragraph{Property} Lets first show that: For $A$ and $B$ two convexe set such that $A \subset B$ and $f$ a function continuous defined on $B$ such that $f$ is $\lambda_A$ lipschitz on $A$ and $\lambda_B$ lipschitz on $B - A$ then $f$ is $\max(\lambda_A, \lambda_B)$ lipshitz on $B$.

\begin{itemize}
	\item $\forall x, y \in A $ : $\left| f(x) - f(y) \right| \le \lambda_A \left\Vert x - y \right\Vert \le \max(\lambda_A, \lambda_B) \left\Vert x - y \right\Vert$
	\item $\forall x, y \in B-A $ : $\left| f(x) - f(y) \right| \le \lambda_B \left\Vert x - y \right\Vert \le \max(\lambda_A, \lambda_B) \left\Vert x - y \right\Vert$
	\item $\forall x \in A$ and $\forall y \in B-A$ : $\exists t \in \left[ 0, 1\right]$ such that $y' = tx + (1-t)y \in \overline{B-A} \cap A$ ($y'$ is on the segment $\left[ x, y\right]$ and at the frontiere between $A$ and $B-A$). Then $\left\Vert x - y \right\Vert = \left\Vert x - y' \right\Vert + \left\Vert y' - y \right\Vert$. 
	\newline We have: $\left| f(x) - f(y) \right| = \left| f(x) -f(y') + f(y') - f(y) \right| \le \left| f(x) -f(y') \right| + \left| f(y') - f(y) \right| \le \lambda_A \left\Vert x - y' \right\Vert + \lambda_B \left\Vert y' - y \right\Vert \le \max(\lambda_A, \lambda_B) \left\Vert x - y' \right\Vert + \max(\lambda_A, \lambda_B) \left\Vert y' - y \right\Vert = \max(\lambda_A, \lambda_B) \left\Vert x - y \right\Vert$
\end{itemize}

In every case $f$ is $\max(\lambda_A, \lambda_B)$ lipchitz on $B$.

\paragraph{$\alpha$ is $\frac{1}{(k-1)r}$ lipschitz:}

\begin{itemize}
	\item $\forall x, y \in B(c, rk) - B(c, r)$ : $\left\Vert \alpha(x) - \alpha(y) \right\Vert = \frac{\left\Vert x-c \right\Vert - \left\Vert y - c \right\Vert}{(k-1)r} \le \frac{\left\Vert x-y \right\Vert}{(k-1)r}$ (because of the triangular inequality)
	\item $\forall x, y \in B(c, r)$ : $\left\Vert \alpha(x) - \alpha(y) \right\Vert = \left\Vert 0 - 0 \right\Vert = 0$
	\item $\forall x, y \in \mathbb{R}^3 - B(c, r)$ : $\left\Vert \alpha(x) - \alpha(y) \right\Vert = \left\Vert 1 - 1 \right\Vert = 0$
		
	\item $\forall x \in B(c, r)$ and $\forall y \in B(c, rk) - B(c, r)$ : Because $B(c, rk)$ and $B(c, r)$ are convexe, by appling the property, we have : $\left| \alpha(x) - \alpha(y) \right| \le \max(0, \frac{1}{(k-1)r})\left\Vert x-y \right\Vert = \frac{\left\Vert x-y \right\Vert}{(k-1)r}$

	\item $\forall x \in B(c, kr)$ and $\forall y \in \mathbb{R}^3 - B(c, kr)$ by using the same argument, we have : $\left| \alpha(x) - \alpha(y) \right| \le \frac{\left\Vert x-y \right\Vert}{(k-1)r}$



\end{itemize}
In evry case, $\alpha$ is $\frac{1}{(k-1)r}$ lipschitz.

\paragraph{$f_k$ is $1 + \frac{k+2}{k-1}$ lipschitz} We have $f_k(p) = (1-\alpha_k(p))f(p) + \alpha_k(p)f_S(p)$ with
\begin{itemize}
	\item $\alpha_k$ : $\mathbb{R}^3 \to \left[0, 1\right]$, $\lambda_\alpha$ lipschitz.
	\item $f$ : $\mathbb{R}^3 \to \mathbb{R}$, $\lambda_1$ lipschitz.
	\item $f_S$ : $\mathbb{R}^3 \to \mathbb{R}$, $\lambda_2$ lipschitz.
\end{itemize}

Then, $\forall x, y \in \mathbb{R}^3$:


\begin{align*}
	\left| f_k(x) - f_k(y) \right| &= \left| (1-\alpha_k(x))f(x) + \alpha_k(x)f_S(x) - (1-\alpha_k(y))f(y) - \alpha_k(y)f_S(y) \right| \\
	&= | \alpha_k(x)f_S(x) ~~- \alpha_k(x)f_S(y) + \alpha_k(x)f_S(y) ~~- \alpha_k(y)f_S(y)  \\
	&~~~~+  (1-\alpha_k(x))f(x) ~~- (1-\alpha_k(x))f(y) + (1-\alpha_k(x))f(y) - (1-\alpha_k(y))f(y) | \\
	&= | \alpha_k(x)[f_S(x) - f_S(y)] + f_S(y)[\alpha_k(x) - \alpha_k(y)]  \\
	&~~~~+  (1-\alpha_k(x))[f(x) -f(y)] + f(y)[(1-\alpha_k(x)) - (1-\alpha_k(y))] | \\ \\
	 \texttt{ because}~~& f(y)[(1-\alpha_k(x)) - (1-\alpha_k(y))] = -f(y)(\alpha_k(x) - \alpha_k(y)) \\ \\
	&= \left| \alpha_k(x)(f_S(x) - f_S(y)) + (1-\alpha_k(x))(f(x) -f(y)) + (f_S(y) - f(y))(\alpha_k(x) - \alpha_k(y)) \right| \\
	& \le \left| \alpha_k(x)(f_S(x) - f_S(y)) \right| + \left| (1-\alpha_k(x))(f(x) -f(y)) \right| + \left| (f_S(y) - f(y))(\alpha_k(x) - \alpha_k(y)) \right| \\
	& = \alpha_k(x)\left|f_S(x) - f_S(y) \right| +  (1-\alpha_k(x))\left|f(x) -f(y) \right| + \left| (f_S(y) - f(y))(\alpha_k(x) - \alpha_k(y)) \right| \\
	& \le \alpha_k(x)\lambda_2 \left\Vert x - y \right\Vert +  (1-\alpha_k(x))\lambda_1 \left\Vert x - y \right\Vert + \left| (f_S(y) - f(y))\right|\lambda_\alpha \left\Vert x - y \right\Vert \\
	& \le \left[\max(\lambda_1, \lambda_2) + \lambda_\alpha \sup_{y}(\left| f_S(y) - f(y)\right|) \right]\left\Vert x - y \right\Vert \\
\end{align*}


On $B(c, kr)$ we have: $\lambda_1 = \lambda_2 = 1$, $\lambda_\alpha = \frac{1}{(k-1)r}$.
Furthemore:
\begin{itemize}
	\item $\lambda_1 = \lambda_2 = 1$
	\item $\lambda_\alpha = \frac{1}{(k-1)r}$
	\item 
		\begin{list}{$\circ$}{}  
		\item $ -r \le f(y) \le (k+1)r$ 
		\item $ -r \le f_S(y) \le (k-1)r$
		\end{list}
		which imply that:
		\begin{list}{$\circ$}{}  
			\item $ -(k+2)r = -(k+1)r - r \le f_S(y)-f(y) \le r + (k-1)r = kr$
			\item $ -kr = -(k-1)r - r \le f(y)-f_S(y) \le r + (k+1)r = (k+2)r$
		\end{list}
		Which led to : $\sup_{y}(\left| f_S(y) - f(y)\right|) \le (k+2)r$
\end{itemize}
\paragraph{} We finnaly have:
\begin{itemize}
	\item $\forall x, y \in B(c, kr)$ : $\left| f_k(x) - f_k(y) \right| \le (1 + \frac{k+2}{(k-1)})\left\Vert x - y \right\Vert $
	\item $\forall x, y \in \mathbb{R}^3 - B(c, kr)$ : $\left| f_k(x) - f_k(y) \right| = \left| f_S(x) - f_S(y) \right| \le \left\Vert x - y \right\Vert$
	\item Because $B(c, kr)$ and $\mathbb{R}^3$ are convexe, $\forall x \in B(c, kr)$ and $\forall y \in \mathbb{R}^3 - B(c, kr)$ : $\left| f_k(x) - f_k(y) \right| \le \max(1 + \frac{k+2}{k-1}, 1) \left\Vert x - y \right\Vert = (1 + \frac{k+2}{k-1})\left\Vert x - y \right\Vert$
\end{itemize}
In every case, we have shown that $f_k$ is $1 + \frac{k+2}{k-1}$ lipschitz.


\subsection*{Surface noise details}

\paragraph{} I followed the implementation of the paper\cite{zanni2013modelisation} with some adjustment:
\subparagraph{} The base of the algorithm consist on :
\begin{enumerate}
	\item Compute a local frame.
	\item Projecting a point on the surface (following the gradient direction or an arbitrary direction in the local frame).
	\item Computing another local frame at the surface.
	\item Generating random points following a poisson process.
	\item Projecting those random points in the surface local frame.
	\item Evaluate the Gabor kernel for each of the projected point and sum up the results.
	\item Add the noise to the initial SDB.
\end{enumerate}


My adjustments:
\begin{itemize}
	\item Because the gabor kernel has an infinite support, to compute the noise, the original paper\cite{lagae2009procedural} choose to cut the kernel with a radius of $\frac{1}{a}$ ($a$ behind a parameter of the noise). they say that it corresponds to an error of 5\%. I encounter some problems with the normals with this cut, I choose to cut with a radius of $\frac{10}{a}$.
	\item My gabor kernel is normalised with a coefficient going from $1$ to $0$ depending on the distance to the centre of the kernel ($0$ is obtained at a distance of $\frac{10}{a}$).
\end{itemize}


\bibliographystyle{abbrv}
\bibliography{report}


\end{document}