\documentclass[a4paper,12pt]{article}
\usepackage[latin1]{inputenc}
\usepackage[T1]{fontenc}
\usepackage{lmodern}
%\usepackage[frenchb]{babel}
\usepackage{fancyhdr}
\usepackage[top=3cm, bottom=2.5cm, left=2.5cm, right=2.5cm]{geometry}
\usepackage{tabto}
\usepackage{eurosym}%pour le oe
\usepackage{amsmath}%pour les matrices
\usepackage{amsfonts}
\usepackage{graphicx}
\usepackage{subfigure}



\usepackage{titling}
\date{\today}




\fancyfoot[L]{\textbf{Pierre Hubert-Brierre}}
\fancyfoot[C]{\textbf{Fac Lyon 1}}
\fancyfoot[R]{\textbf{\thedate~~~~\thepage}}

\begin{document}

\pagestyle{fancy}
\begin{center}
\textbf{\huge Report virtual world generation}
\end{center}

\pagestyle{fancy}

\begin{figure}[!h]
  \centering
  \includegraphics[scale=0.2]{full_scene.png}{}
\end{figure}

\section*{Utilitary}

\paragraph{} I can export several scalar field from my terrain:
\begin{itemize}
	\item Slopes
	\item Occlusion
	\item Drain
	\item Derivate (on x and y)
	\item Laplacian
\end{itemize}

Here is the results:
\begin{figure}[!h]
  \centering
  \includegraphics[scale=0.2]{full_scene.png}{}
  \caption{a caption}
\end{figure}

\section*{Erosion}

\paragraph{} I implemented the smooth hill erosion and a rain erosion using the equation $...$. After what I can add colors to help visualize the terrain, puting water bellow the height $0$ and at every point where the drain is greater than a certain threshold (to have rivers). I then add grass where the slope is small enought and rock every where else.
\begin{figure}[!h]
  \centering
  \includegraphics[scale=0.2]{full_scene.png}{}
  \caption{raw terrain, smoothed terrain, eroded terrain, colored terrain}
\end{figure}

\section*{Civilisation}

\paragraph{} I used Dijkstra for shortest path computation and geometry graphs for network generation.

\subsection*{Roads between cities}

\paragraph{} Given a set of position (the cities locations), I compute the shortest path between each pair of cities (using Dijkstra algorithm on a weigthed network (weigths are computing depending on the terrain specification (slope, size, water...))) and then, I remove useless path (a path between $A$ and $B$ is useless if, for a fixes real $\alpha$, there exists another point $C$ such that $\left| AB \right|^\alpha > \left| AC \right|^\alpha + \left| CB \right|^\alpha$). To get a road network linking every cities.

\subsection*{Streets inside cities}

\paragraph{} To generate cities, I used a Eden growth algorithm to create crossroads. After what, I simply link them together in the same way as I linked cities. Finaly, I walk along each street, adding a house if possible (If the house does not intersect a street or another house, the intersection computation is done using a Bounding Volume Hierarchy to speed up the process).

\subsection*{results}

\begin{figure}[!h]
  \centering
  \includegraphics[scale=0.2]{full_scene.png}{}
  \caption{raw terrain, smoothed terrain and eroded terrain}
\end{figure}

\section*{Nature}

\paragraph{} Finaly, I can generate forest of different tree type. In order to do that, I precompute several forest tiles using a drath throwing algorithm. Of course, I make sure that I can put every tile next to each other. Once the set of tile generated, I simply pave the terrain with forest tile and create a tree at each tree location of each tiles.

\subparagraph{} To create a tree, I compute a probability of creation for every tree types. Then I draw a random number for each tree types to know if it can be created or not.
\begin{itemize}
	\item If no tree is created, I do nothing
	\item if one tree type can be created, I create it (instanciate the tree model with a random size proportional to the probability of creation)
	\item If more than two tree can be created, I select one tree type randomly (see ANEXES).
\end{itemize}

\paragraph{} The probability of creation is $0$ if there is some water, road or house in the tree location. This is computed using the same intersection algorithm as the houses intersection (SEE ANEXES).

\section*{Anexes}

TODO



\end{document}